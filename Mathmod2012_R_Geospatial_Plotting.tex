%===============================================================================
% $Id: ifacconf.tex 19 2011-10-27 09:32:13Z jpuente $  
% Template for IFAC meeting papers
% Copyright (c) 2007-2008 International Federation of Automatic Control
%===============================================================================
\documentclass{ifacconf}

\usepackage{graphicx}      % include this line if your document contains figures
\usepackage{natbib}        % required for bibliography
%===============================================================================
\begin{document}
\begin{frontmatter}

\title{R \& GIS: Geospatial Plotting} 
% Title, preferably not more than 10 words.

\thanks[footnoteinfo]{Sponsor and financial support acknowledgment
goes here. Paper titles should be written in uppercase and lowercase
letters, not all uppercase.}

\author[First]{Florian Endel} 
\author[Second]{Peter Filzmoser} 

\address[First]{FFG IFEDH project, Student at Vienna University of Technology (e-mail: florian@endel.at).}
\address[Second]{Department of Statistics and Probability Theory,
Vienna University of Technology (e-mail: P.Filzmoser@tuwien.ac.at)}

\begin{abstract}                % Abstract of not more than 250 words.
Examples of spatial data within the R environment and the combination of R with data sets, spatial
tools, libraries and other software products, which are common in real life environments, are provided in this paper.

Beginning with the setup of a new project using Git and an online repository, a Document build by Sweave - a
combination of R and \LaTeX{} - is explained. Once the fundamental setup is working rudimentarily, the import
of data and (geo-) spatial information from different sources is shown. Finnaly some examples of spatial plots
using the \textit{sp} package in R are  included.

Additionally some tools like "integrated development environments" (IDE), 
which may be supportive during the daily work
and help newcomers learning the presented techniques, 
are mentioned and specific programs are recommended.
\end{abstract}

\begin{keyword}
R, Sweave, \LaTeX{}, Spatial Data, Git
\end{keyword}

\end{frontmatter}
%===============================================================================

\section{Introduction}
Building a sophisticated report including spatial data, plots and usefule information
(normally) takes some effort. 

\section{Setup of the project}

Git

\section{Document Structure}

LaTeX, Sweave

\section{About Data}

getting Data from files, DB and online resources


\section{Geospatial Plotting}

examples of Geospatial Plots

\section{Conclusion}



\bibliography{lit}            


\end{document}
